%Copyright (C) 2018 Matevž Markovič (matevz.markovic.v@gmail.com)
%You are free to redistribute this asss much as you please.

\def \MalnetoClanekTemplateVersion{1.21}

%--------------------------------------------------------------------
%--------------------------------------------------------------------
%--------------------------------------------------------------------

%Project
%\def \projekt{Knjiga totaal}
%\def \projektK{KT}
\def \projektKb{MM}

%Title
\def \naslov{Dvajset pomembnih spiritualnih navodil}

%Month year
\def \monthYear{Junij 2018}
\def \year{2018}

%ID within the project
\def \MalnetoID{11}
\def \MMid{101}

%Build version
\def \buildversion{1}

%ID of the work, prefixed by "KT_", preferably numerical (for example, RID_4.3)
%To gre na zadnjo stran k tehničnim podatkom!
\def \IDworkMalneto{\projektKa\_\MalnetoID$.$\buildversion}         	%***********Unique!!!*************
\def \IDworkMM{\projektKb\_\MMid$.$\buildversion}

%Author of the work
\def \avtor{Matev\v{z} Markovi\v{c}}
\def \avtorEmail{\href{mailto:matevz.markovic.v@gmail.com}{matevz.markovic.v@gmail.com}}

%--------------------------------------------------------------------
%--------------------------------------------------------------------
%--------------------------------------------------------------------

\documentclass[12pt,a4paper]{article}

%Import the settings
%Copyright (C) 2015 Matevž Markovič (matevz.markovic@guest.arnes.si). All rights reserved.

%Parameters

\usepackage{verse}

%-----------------------

\usepackage{url}
\usepackage[utf8]{inputenc}
\usepackage[T1]{fontenc}
\usepackage{amsthm}
\usepackage{amsmath}
\usepackage[slovene]{babel} 			%V Slovenščini?

%Za v RID-TP \usepackage[english,slovene]{babel}

\usepackage{amssymb}
\usepackage{graphicx}
\usepackage[pdftex,colorlinks,citecolor=black,filecolor=black,linkcolor=black,urlcolor=black,pagebackref]{hyperref}
\usepackage{enumerate}
\usepackage{framed}
\usepackage[protrusion=true,expansion=true]{microtype}
\usepackage{fancyhdr}                       %noge in glave
\usepackage{lipsum}                         %random tekst
\usepackage[usenames,dvipsnames]{color}     %Za barve (https://en.wikibooks.org/wiki/LaTeX/Colors)

%\usepackage{fullpage}
\usepackage{placeins}
\usepackage{array}
\usepackage{multirow}

%Parametri dokumenta (po predlogi diplomske na FRI) - NE UPOŠTEVAJ!
%\oddsidemargin 1.4cm
%\evensidemargin 0.35cm
%\textwidth 14cm
%\topmargin 0.26cm
%\headheight 0.6cm
%\headsep 1.5cm
%\textheight 20cm


%https://en.wikibooks.org/wiki/LaTeX/Page_Layout
%\setlength{\topmargin}{-.5in}
%\setlength{\textheight}{9in}
%\setlength{\oddsidemargin}{.125in}
%\setlength{\textwidth}{6.25in}
\usepackage{a4wide}

%Theorems
\theoremstyle{definition}
\newtheorem{Definition}{Definition}
\newtheorem{Axiom}{Axiom}
\theoremstyle{plain}
\newtheorem{Theorem}{Theorem}
%A proposition is a generic term for a theorem of no particular importance.
\newtheorem{Proposition}{Proposition}
%A lemma is a "helping theorem", a proposition with little applicability except that it forms part of the proof of a larger theorem.
\newtheorem{Lemma}{Lemma}        
%A corollary is a proposition that follows with little or no proof from one other theorem or definition.                
\newtheorem{Corollary}{Corollary}
\newtheorem{Proof}{Proof}
\theoremstyle{remark}
\newtheorem{Remark}{Remark}
\newtheorem{Idea}{Idea}

%
\hypersetup{colorlinks}% uncomment this line if you prefer colored hyperlinks (e.g., for onscreen viewing)

%Iz template-a za ZZRS
% fixed width column with centering (  ex. C{2cm}  )
\newcolumntype{C}[1]{>{\centering\arraybackslash}p{#1}}

% table row padding
\renewcommand{\arraystretch}{1.5}

% enumerate subsubsections and include them in TOC
\setcounter{secnumdepth}{3}
\setcounter{tocdepth}{3}

%++++++++++++++++++++++++++++++++++++++++++++++++++++++
%++++++++++++++++++++++++++++++++++++++++++++++++++++++
%+++++       CHANGES IN THE RID+ PROJECT         ++++++
%++++++++++++++++++++++++++++++++++++++++++++++++++++++
%++++++++++++++++++++++++++++++++++++++++++++++++++++++

%Define the abstract page
\newenvironment{abstractpage}
  {\cleardoublepage\vspace*{\fill}\thispagestyle{empty}}
  {\vfill\cleardoublepage}
\renewenvironment{abstract}[1]
  {\bigskip\selectlanguage{#1}%
   \begin{center}\bfseries\abstractname\end{center}}
  {\par\bigskip}

\usepackage{lettrine} % The lettrine is the first enlarged letter at the beginning of the text
%\usepackage{paralist} % Used for the compactitem environment which makes bullet points with less space between them

\usepackage{abstract} % Allows abstract customization
\renewcommand{\abstractnamefont}{\normalfont\bfseries} % Set the "Abstract" text to bold
\renewcommand{\abstracttextfont}{\normalfont\small\itshape} % Set the abstract itself to small italic text
\usepackage{tcolorbox}

%1.1 slo changes
\usepackage{enumitem}

%+++++++++++++++++++++++++++++++++++++++++++++++++++++
%+++++  CHANGES IN THE KNJIGA TOTAAL PROJECT    ++++++
%+++++++++++++++++++++++++++++++++++++++++++++++++++++
\usepackage{wrapfig}

%Predloga 1.2 dodatek
%---------------------

%http://tex.stackexchange.com/questions/25249/how-do-i-use-a-particular-font-for-a-small-section-of-text-in-my-document
\newenvironment{mPisava}{\fontfamily{pzc}\selectfont\color{blue}}{\par}
\DeclareTextFontCommand{\metaPisava}{\mPisava}
%Sprejemljive pisave --> ccr, phv, pnc, ppl, ptm, put, pzc (pisanja, privzeta)
%---------------------

\usepackage{xcolor}



%--------------------------------------------------------------------------------------------
%	FANCY HEADER SETTINGS SECTION
%--------------------------------------------------------------------------------------------
\fancypagestyle{plain}
{
    \fancyhead{} % Blank out the default header
    \fancyfoot{} % Blank out the default footer
    %\fancyhead[C]{\naslov $\bullet$ Project \projektK $\bullet$ May 2015} % Custom header text
    %\fancyhead[L]{\textbf{\naslov} $\subseteq$ \textbf{Projekt \projektK}}
    \fancyhead[L]{\textbf{\naslov} (\projektKb)}
    \fancyhead[R]{\monthYear}
    
    %1.1 template - change (spaces ';')
    \fancyfoot[L]{\copyright\ \year\  \avtor}
    \fancyfoot[R]{\thepage} % Custom footer text
}
\pagestyle{plain}
%---------------------------------------------------------------------------------------------

%--------------------------------------------------------------------------------------------
%	TITLE SECTION
%--------------------------------------------------------------------------------------------
\title{
	\vspace{-15mm}
	%\Large \emph{Project \projekt} \\[2mm]
    %\Large \emph{\projektKb \ article} \\[2mm]
	\fontsize{24pt}{10pt}\textbf{\naslov}	%\\
	%\large ID: \IDwork	\\	
}

\author{
\large
\textsc{\avtor}\thanks{Lahko mi pišete na naslov \avtorEmail.}
%\vspace{-5mm}
}
\date{}
%---------------------------------------------------------------------------------------------

%++++++++++++++++++++++++++++++++++++++++++++++++++++++++++++++++++++++++++++++
%+++++++++++++++++++++++++Start of document++++++++++++++++++++++++++++++++++++
%++++++++++++++++++++++++++++++++++++++++++++++++++++++++++++++++++++++++++++++
\begin{document}

\maketitle % Insert title
\thispagestyle{fancy} % All pages have headers and footers

\begin{abstract}
%	%\noindent \lipsum[1] % Dummy abstract text
    \noindent Ta dokument je prevod serije govorov Dvajset pomembnih spiritualnih navodil Sri Swami Šivanande avtorja Sri Swami Čidanande. Angleški izvirnik je bil objavljen leta 1999 na spletni strani \url{http://www.dlshq.org/}.
\end{abstract}

%>>> http://tex.stackexchange.com/questions/227636/package-fancyhdr-warning-headheight-is-too-small-41-6pt

%+++++++++++++++++++++++++++++++++++++++++++++++++++++++++++++++++++++++++++++++++++
%+++++++++++++++++++++++++++++++++++++++++++++++++++++++++++++++++++++++++++++++++++
%+++++++++++++++++++++++++++++++++++++++++++++++++++++++++++++++++++++++++++++++++++

\tableofcontents

\newpage

%++++++++++++++++++++++++++++++++
%!!!!!!!!!!!!!!!!!!!!!!!!!!!!!!!!!!!!!
%\part{Confronting_the_fear}
%\label{part:confronting_the_fear}
%\chapter{Confronting the fear}
%\input{./text/confronting_the_fear.tex}
%*+++++++++++++++++++++++++++++++

%%++++++++++++++++++++++++++++++++
%\input{./text/text.tex}
%Copyright (C) 2017 Matevž Markovič (matevz.markovic@guest.arnes.si) [All rights reserved.]

%http://texblog.org/2011/05/21/drop-caps-with-lettrine/
%http://mirror.math.ku.edu/tex-archive/macros/latex/contrib/lettrine/doc/lettrine.pdf
\vspace{5mm}
%\lettrine[nindent=0em,lines=1]{S}{ledeči} stavek se mora vleči skozi več vrtic, sicer bo problem, saj bo sicer problem, ker bo prekrivanje, pa bo potem štala, in bodo vsi jezni, nekateri pa še bolj, zato bolje da gre skozi več vrstic, in bom zadovoljen.

%\section{Uvod}
%\lipsum[2]


%1.1 - change in the quote frame
%[background=color,backgroundcolor=gray] ; http://wiki.contextgarden.net/Framed
%\definecolor{shadecolor}{gray}{0.8}
%\begin{centering}
%    %\begin{framed}
%        \begin{quote}
%        \emph{``I haven't lost my mind; I have a tape back-up somewhere.''}
%
%        \begin{flushright}
%        --- Author Unknown
%        \end{flushright}
%        
%        \end{quote}
%    %\end{framed}
%\end{centering}

%Predloga 1.2 dodatek
%---------------------
%\metaPisava{Komentar vsebine se začne tu\ldots \emph{žđžćč}\lipsum[1] To če biti to! \ldots tu pa se konča komentar vsebine.}
%---------------------

%\lipsum[3]

%\section{Odvod}

%\lipsum

%Invoke me with \GTtechnique{}
%\newcommand{\AlfaGibanje}{$\hat{\alpha}$-gibanje}

\section*{Opombe k prevodu}
Prevodi terminov iz Sanskrita so bili povzeti iz Malega slovarja azijskih filozofij Maje Milčinski, izdanega s strani Filozofskega inštituta ZRC SAZU leta 2003 v Ljubljani.

\emph{Om Namo Bhagavate Šivanandaya!}

\section*{Predgovor}
%A life without spiritual sadhana is a dreary waste. A life with spiritual sadhana is wise living—a life that will lead to Blessedness. The combination, the blending together, the harmonising, the combining of an active inner spiritual life with an active outer secular life, fulfilling of legitimate duties and obligations—unavoidable, inevitable actions—this combining of the spiritual with the secular, the Divine with the earthly was Gurudev’s special mission.

% And to that end He gave us one aspect of His teachings in the form of the 20 Important Spiritual Instructions. 

%These instructions comprise a harmonising of the outer and the inner, the emphasising of the inner and practising of the teachings of the ancient sages and seers of the Upanishads, practising of the teachings of all the saints who through the centuries have graced and blessed Bharatavarsha by their ideal life and teachings. This practice, together with the normal life is Gurudev’s special teaching to the world, and summing up, the practical 20 Important Spiritual Instructions give us the key to Blessedness even while living in the world and through the world itself. 

\section*{Predgovor}
Tako kot je življenje brez spiritualne s\={a}dh\={a}ne\footnote{S\={a}dh\={a}na je duhovna praksa, katere cilj je osvoboditev.} turobna potrata, je življenje s spiritualno s\={a}dh\={a}no modro. Takšno življenje brez dvoma vodi v Blagoslovljenost.
Združevanje spiritualnega ter sekularnega je bila Gurudevova\footnote{Gurudev ali \emph{guru} je duhovni učitelj. \emph{Gu} označuje temo (ignoranco), \emph{ru} pa označuje spiritualno znanje ki prežene to temo. Guru je tako tisti, ki prežene spiritualno temo oziroma spiritualno ignoranco.} posebna naloga na naši Zemlji.
Vsemu svetu je dal vzor kombinacije, harmonizacije aktivnega notranjega spiritualnega življenja z aktivnim zunanjim sekularnim življenjem, v katerem se neprenehno opravlja vse zunanje dolžnosti in obveznosti - življenje neizogibnega, neizbežnega ter nenehnega delovanje. Teh dvajset spiritualnih navodil predstavlja pomemben vidik Njegovih naukov, ki nam omogočajo harmonizacijo notranjega ter zunanjega, kot tudi pospešen razvoj notranje spiritualnost. Navezujejo se na navodila starodavnih modrecev ter vidcev iz Upanišad\footnote{Upani\d{s}ade so zadnji del Ved, ki filozofsko povzemajo njihova učenja. Njihova osrednja tema je vselej pomen Atmana (duše, čiste zavesti; večnega, nespremenljivega ter nedotakljivega jedra posameznika) ter Boga Brahmana (večne biti vseh stvari \emph{Sat}, popolne zavesti \emph{Čit} in neizmerne blaženosti \emph{Ananada}), spoznanje njune identičnosti ter pomen svetega zloga Om.}, kot tudi na nauke vseh svetnikov, ki so skozi stoletja blagoslovljali Bharatavar\d{s}o\footnote{Bh\={a}ratavar\d{s}a v neposrednem pomenu označuje kontinent, ki je posvečen luči, modrosti. Lahko označuje le Indijo (uradno ime Indije je \emph{Bh\={a}rata}), glede na ep Mah\={a}bh\={a}rato pa zaobsega vsa ozemlja, katerim je vladal kralj Bharata: Indija, Baktrija, Uzbekistan, Afganistan, Tadžikistan,  Kirgizistan, Turkmenistan ter deli Tibeta. Lahko označuje tudi celo Zemljo.} z njihovim idealnim življenjem in nauki.
Praksa teh dvajsetih navodil v vsakdanjem življenju je Gurudevovo posebno navodilo svetu. Preko njih dobimo ključ do stanja Blagoslovljenosti še za časa življenja v tem svetu.


%%++++++++++++++++++++++++++++++++

%+++++++++++++++++++++++++++++++++++++++++++++++++++++++++++++++++++++++++++++++++++
%+++++++++++++++++++++++++++++++++++++++++++++++++++++++++++++++++++++++++++++++++++
%+++++++++++++++++++++++++++++++++++++++++++++++++++++++++++++++++++++++++++++++++++

%\bibliography{matevz_markovic_tango_practice_bibliography}
%\bibliographystyle{plain}

%\cite{*} odstrani to!

%\definecolor{shadecolor}{rgb}{0.64,0.87,0.93}  %http://latexcolor.com/ ; Non-photo blue
%\begin{shaded}
%http://tex.stackexchange.com/questions/66154/how-to-construct-a-coloured-box-with-rounded-corners

%\vspace{5mm}
%\newtcolorbox{metaPodatki}[1]{fonttitle=\bfseries,title=#1}
%\begin{metaPodatki}{Metadata}
%
%Namen mojih člankov je vaja veščine pisanja ter vzpodbujanje razmišljanja, saj najraje razmišljam prav ob pisanju. Vsakemu mojemu članku pripada enoličen niz znakov, preko katerega se nanj lahko sklicujemo. %// Dom mojih člankov je spletna stran \emph{\spletnaStran}.\\
%

%Technical data about this article:
%1.1 template - change ('label={}'); add package 'enumitem' to parameters.tex
%\begin{itemize}[label={}]
%    \item it can be as a general (MM) article uniquely identified by the string \emph{\IDworkMM}, where \emph{\MMid} is the identification number of the article and \emph{\buildversion} version of the article;
%    \item last compiled on \today;
%    \item written upon the template version \emph{\MalnetoClanekTemplateVersion}.
%\end{itemize}
%
%\end{metaPodatki}

\end{document} 
%++++++++++++++++++++++++++++++++++++++++++++++++++++++++++++++++++++++++++++++
%+++++++++++++++++++++++++++End of document++++++++++++++++++++++++++++++++++++
%++++++++++++++++++++++++++++++++++++++++++++++++++++++++++++++++++++++++++++++

%+++++++++++++++++++ INSTRUCTIONS +++++++++++++++++++++++
%pdflatex -output-directory ./latex_output sample2.tex

%%Compilation sequence (bibtex):

%0.From the root directory
%1.  pdflatex -output-directory ./latex_output 20navodil.tex
%2.  bibtex latex_output/20navodil.aux
%3.  pdflatex -output-directory ./latex_output 20navodil.tex
%4.  pdflatex -output-directory ./latex_output 20navodil.tex

%Artha offline dictionary would be excellent... and indeed it is
%++++++++++++++++++++++++++++++++++++++++++++++++++++++++

%Bibtex
%@Misc{MMclanek,
%    author = 	 {Matevž Markovič},
%    title = 	 {clanek},
%    howpublished = {Članek je na voljo preko naslova \url{http://www2.arnes.si/~mmarko7}},
%    month = 	 {Avgust},
%    year = 	 {2015},
%    note = 	 {ID članka je \emph{MM_1$.$11}}
%}
%
