%Copyright (C) 2015 Matevž Markovič (matevz.markovic@guest.arnes.si). All rights reserved.

%Parameters

\usepackage{url}
\usepackage[utf8]{inputenc}
\usepackage[T1]{fontenc}
\usepackage{amsthm}
\usepackage{amsmath}
\usepackage[slovene]{babel} 			%V Slovenščini?

%Za v RID-TP \usepackage[english,slovene]{babel}

\usepackage{amssymb}
\usepackage{graphicx}
\usepackage[pdftex,colorlinks,citecolor=black,filecolor=black,linkcolor=black,urlcolor=black,pagebackref]{hyperref}
\usepackage{enumerate}
\usepackage{framed}
\usepackage[protrusion=true,expansion=true]{microtype}
\usepackage{fancyhdr}                       %noge in glave
\usepackage{lipsum}                         %random tekst
\usepackage[usenames,dvipsnames]{color}     %Za barve (https://en.wikibooks.org/wiki/LaTeX/Colors)

%\usepackage{fullpage}
\usepackage{placeins}
\usepackage{array}
\usepackage{multirow}

%Parametri dokumenta (po predlogi diplomske na FRI) - NE UPOŠTEVAJ!
%\oddsidemargin 1.4cm
%\evensidemargin 0.35cm
%\textwidth 14cm
%\topmargin 0.26cm
%\headheight 0.6cm
%\headsep 1.5cm
%\textheight 20cm


%https://en.wikibooks.org/wiki/LaTeX/Page_Layout
%\setlength{\topmargin}{-.5in}
%\setlength{\textheight}{9in}
%\setlength{\oddsidemargin}{.125in}
%\setlength{\textwidth}{6.25in}
\usepackage{a4wide}

%Theorems
\theoremstyle{definition}
\newtheorem{Definition}{Definition}
\newtheorem{Axiom}{Axiom}
\theoremstyle{plain}
\newtheorem{Theorem}{Theorem}
%A proposition is a generic term for a theorem of no particular importance.
\newtheorem{Proposition}{Proposition}
%A lemma is a "helping theorem", a proposition with little applicability except that it forms part of the proof of a larger theorem.
\newtheorem{Lemma}{Lemma}        
%A corollary is a proposition that follows with little or no proof from one other theorem or definition.                
\newtheorem{Corollary}{Corollary}
\newtheorem{Proof}{Proof}
\theoremstyle{remark}
\newtheorem{Remark}{Remark}
\newtheorem{Idea}{Idea}

%
\hypersetup{colorlinks}% uncomment this line if you prefer colored hyperlinks (e.g., for onscreen viewing)

%Iz template-a za ZZRS
% fixed width column with centering (  ex. C{2cm}  )
\newcolumntype{C}[1]{>{\centering\arraybackslash}p{#1}}

% table row padding
\renewcommand{\arraystretch}{1.5}

% enumerate subsubsections and include them in TOC
\setcounter{secnumdepth}{3}
\setcounter{tocdepth}{3}

%++++++++++++++++++++++++++++++++++++++++++++++++++++++
%++++++++++++++++++++++++++++++++++++++++++++++++++++++
%+++++       CHANGES IN THE RID+ PROJECT         ++++++
%++++++++++++++++++++++++++++++++++++++++++++++++++++++
%++++++++++++++++++++++++++++++++++++++++++++++++++++++

%Define the abstract page
\newenvironment{abstractpage}
  {\cleardoublepage\vspace*{\fill}\thispagestyle{empty}}
  {\vfill\cleardoublepage}
\renewenvironment{abstract}[1]
  {\bigskip\selectlanguage{#1}%
   \begin{center}\bfseries\abstractname\end{center}}
  {\par\bigskip}

\usepackage{lettrine} % The lettrine is the first enlarged letter at the beginning of the text
%\usepackage{paralist} % Used for the compactitem environment which makes bullet points with less space between them

\usepackage{abstract} % Allows abstract customization
\renewcommand{\abstractnamefont}{\normalfont\bfseries} % Set the "Abstract" text to bold
\renewcommand{\abstracttextfont}{\normalfont\small\itshape} % Set the abstract itself to small italic text
\usepackage{tcolorbox}

%1.1 slo changes
\usepackage{enumitem}

%+++++++++++++++++++++++++++++++++++++++++++++++++++++
%+++++  CHANGES IN THE KNJIGA TOTAAL PROJECT    ++++++
%+++++++++++++++++++++++++++++++++++++++++++++++++++++
\usepackage{wrapfig}

%Predloga 1.2 dodatek
%---------------------

%http://tex.stackexchange.com/questions/25249/how-do-i-use-a-particular-font-for-a-small-section-of-text-in-my-document
\newenvironment{mPisava}{\fontfamily{pzc}\selectfont\color{blue}}{\par}
\DeclareTextFontCommand{\metaPisava}{\mPisava}
%Sprejemljive pisave --> ccr, phv, pnc, ppl, ptm, put, pzc (pisanja, privzeta)
%---------------------

\usepackage{xcolor}
