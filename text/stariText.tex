%Copyright (C) 2015 Matevž Markovič (matevz.markovic@guest.arnes.si) [All rights reserved.]

%http://texblog.org/2011/05/21/drop-caps-with-lettrine/
%http://mirror.math.ku.edu/tex-archive/macros/latex/contrib/lettrine/doc/lettrine.pdf
\vspace{5mm}
%\lettrine[nindent=0em,lines=1]{S}{ledeči} stavek se mora vleči skozi več vrtic, sicer bo problem, saj bo sicer problem, ker bo prekrivanje, pa bo potem štala, in bodo vsi jezni, nekateri pa še bolj, zato bolje da gre skozi več vrstic, in bom zadovoljen.

%\section{Uvod}
%\lipsum[2]


%1.1 - change in the quote frame
%[background=color,backgroundcolor=gray] ; http://wiki.contextgarden.net/Framed
%\definecolor{shadecolor}{gray}{0.8}
%\begin{centering}
%    %\begin{framed}
%        \begin{quote}
%        \emph{``I haven't lost my mind; I have a tape back-up somewhere.''}
%
%        \begin{flushright}
%        --- Author Unknown
%        \end{flushright}
%        
%        \end{quote}
%    %\end{framed}
%\end{centering}

%Predloga 1.2 dodatek
%---------------------
%\metaPisava{Komentar vsebine se začne tu\ldots \emph{žđžćč}\lipsum[1] To če biti to! \ldots tu pa se konča komentar vsebine.}
%---------------------

%\lipsum[3]

%\section{Odvod}

%\lipsum
